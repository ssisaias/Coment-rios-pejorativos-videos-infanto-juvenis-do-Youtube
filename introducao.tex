\section{INTRODUÇÃO}


A internet é uma rede mundial que interliga milhões de computadores em todo o mundo, servindo como um grande fator de comunicação e integração social \cite{marioFalcao2015}. Todo os dias, novas páginas na internet surgem, sejam de relacionamento, humor, entretenimento, notícias, ou redes sociais, como exemplo: Facebook, Twitter, Instagram, Google+, Vimeo, DailyMotion e Youtube.

É pertinente que em nossa sociedade contemporânea, as pessoas estejam mais próximas da tecnologia, principalmente as crianças. 
Essas crianças adquirem habilidades diferentes das crianças de antigamente: enquanto uma criança da década de 80 possuía uma maior facilidade para construir ou modelar um brinquedo, as crianças da geração atual possuem habilidades para lidar com a informática, devido ao convívio rotineiro com a mesma \cite{marioFalcao2016}.

% Motivação para utilização do Youtube
A plataforma de vídeos Youtube é o segundo site mais acessado do mundo \cite{alexaYoutube}. Em tal plataforma, é possível postar e compartilhar vídeos com uma enorme e abrangente gama de conteúdos, servindo também como site de buscas. É possível encontrar conteúdo musical, \textit{vlogs}, animações, curta-metragens, conteúdo educativo, conteúdo com público alvo adulto e conteúdo voltado especificamente ao público jovem e infantil. 

Como parte de suas características de rede social, o Youtube disponibiliza, em seus vídeos, a possibilidade de usuários comentarem. Porém tendo em vista as crianças e adolescentes estão conectadas na rede cada vez mais cedo \cite{EnyoGoncalves2017}, ao navegar pela plataforma, em vídeos infantis nota-se que há uma certa liberdade para realizar comentários de qualquer natureza, até mesmo ofensivos ou inapropriados para o público o qual o vídeo é destinado, um exemplo disso pode ser visto na Figura \ref{fig:youtube_comment}.

% Qual a problemática dos comentários negativos em vídeos infantis. Era bom ter referências de artigos para isso. % ainda n achei
% solution -> ECA
Segundo o artigo 17 do \textbf{Estatuto da Criança e do Adolescente} (ECA): "O direito ao respeito consiste na inviolabilidade da integridade física, psíquica e moral da criança e do adolescente, abrangendo a preservação da imagem, da identidade, da autonomia, dos valores, ideias e crenças, dos espaços e objetos pessoais." \cite{eca_lei}.  Assim, é possível através da tecnologia buscar meios para proteger moralmente e psiquicamente o público mais jovem que utiliza a plataforma. %Esse final não ta legal %

\begin{figure}[H] %use h para forçar que a figura fique abaixo do texto
	\caption{\label{fig:youtube_comment} Comentário do Youtube}
	\begin{center}
	    \includegraphics[scale=1]{figuras/figura_1.png} % altere o atributo scale para o tamanho da figura
	\end{center}
	\legend{Fonte: Youtube - 13 de Setembro de 2017}
%Verificar se precisa por a data aqui 
\end{figure}

A Figura \ref{fig:youtube_comment} foi retirada do vídeo \textbf{UPA CAVALINHO - Clipe Música Oficial - Galinha Pintadinha DVD 4}, um vídeo direcionado à crianças de 2 a 7 anos postado pelo canal \emph{Galinha Pintadinha}, que é direcionado ao público infantil. %Explicar o comentário que pode ser visualizado e interpretado pela criança. Também seria bom referências para isso. %N.A. (28nov2017) não encontrei referências para "crianças palavrão"; checar os links abaixo:
%http://onlinelibrary.wiley.com/doi/10.1111/j.1365-2214.2011.01338.x/full
%http://www.tandfonline.com/doi/abs/10.1080/01434630408666529
%procurar tb por children+swear words.


Dessa forma, o problema abordado neste trabalho é identificar comentários pejorativos semelhantes ao apresentado na Figura \ref{fig:youtube_comment} em vídeos infantis postados no Youtube. Para solucionar tal problema, primeiramente será utilizada uma ferramenta própria para coleta dos comentários, seguido da utilização de técnicas de mineração de dados, e técnicas de classificação e análise dos dados. Um modelo será elaborado a fim de apontar os comentários pejorativos dos vídeos infantis, podendo assim, ser utilizado para futuramente elaborar meios de alertar os responsáveis e os pais sobre o nível de segurança dos vídeos que estão sendo exibidos às crianças. O modelo também poderá ser aplicado em qualquer vídeo do Youtube.  % Como tu vai fazer isso? Tem que explicar aqui. Para que? O que tu vai fazer com esses comentários? Soluções para denunciar os comentários.
% Para que e para quem  esse estudo vai servir? Escrever um parágrafo sobre isso.

\textcolor{red}{nada do plugin foi mencionado,pq??}

%Falar das ferramentas que serão utilizadas

%\begin{itemize}
 %   \item Questão de Pesquisa?
  %  \item Questão de Pesquisa? 
   % \item Questão de Pesquisa?
%\end{itemize}

%Neste trabalho o idioma dos textos analisados é estritamente português, e na Seção 5 serão esclarecidos os critérios que são considerados para que um comentário seja classificado como pejorativo.

Este trabalho está estruturado da seguinte forma. No Capítulo 2, serão apresentados os principais trabalhos relacionados, que tratam dos temas de mineração e de proteção de crianças online; No Capítulo 3, são apresentados os objetivos do trabalho; No Capítulo 4, são apresentados os principais conceitos que são utilizados durante o projeto; O Capítulo 5 apresenta a metodologia de trabalho; e o Capítulo 6 apresenta \textcolor{blue}{dados preliminares da pesquisa.}

\textcolor{red}{não deveria ser os resultados?}

\textcolor{red}{mover objetivos e fundamentacao teorica para depois da introdução}
