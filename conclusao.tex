\newpage
\section{CONCLUSÃO E TRABALHOS FUTUROS}
O problema que o trabalho propôs solucionar é o de classificar comentários em vídeos destinados à crianças e adolescentes, tendo em conta a baixa moderação que ocorre nesses comentários, e a incidência de ofensas e termos de baixo calão. 

Através do modelo elaborado foi possível verificar que vídeos destinados especificamente a crianças não possuem um quantidade alta de comentários ofensivos. Segundo a Tabela \ref{resultados-qtd-commentarios-classes}, o vídeo \textbf{PARABÉNS DA GALINHA PINTADINHA - Clipe Música Oficial - Galinha Pintadinha DVD 4} do \textbf{Canal Galinha Pintadinha}, possui aproximadamente 95,56\% comentários neutros e apenas 4,44\% comentários pejorativos.

Por outro lado, vídeos destinados a adolescentes possuem uma taxa bem maior de comentários negativos, tendo em vista que os comentários do vídeo \textbf{Não Faz Sentido! - Crepúsculo [+13]} do \textbf{Canal Felipe Neto} foram classificados com uma taxa de 65,56\% de comentários neutro e 34,43\% de comentários pejorativos.

O modelo não foi amplamente aplicado em vários vídeos, tendo em conta a quantidade de comentários já coletados e o tempo necessário para a realização do trabalho. Mas em contrapartida, a utilização da extensão diminui esse peso, viso que com ela, é possível classificar outros vídeos no Youtube.

Os dados obtidos e o modelo gerado são considerados úteis para a classificação de comentários pejorativos de vídeos no Youtube, mais precisamente vídeos com público alvo adolescente. O que torna relevante o uso da extensão criada para o navegador da web Google Chrome, principalmente quando o uso é direcionado a adolescentes.


Com o objetivo de aumentar a relevância da pesquisa, algumas melhorias podem ser realizadas, principalmente no que diz respeito à aplicação das técnicas utilizadas, assim como na extensão desenvolvida:

\begin{alineas}
    
    \item Aumentar o número de vídeos classificados;
    \item Avaliar o uso da extensão criada, assim como realizar melhorias de UX e trazer para outros navegadores;
    \item Permitir que o público avalie a classificação realizada, tornando classificação mais precisa;
    \item Melhorar o modelo de classificação criado através do \textit{feedback} obtido;
    
\end{alineas}

\newpage