\newpage
\section{CONCLUSÃO}

% \textcolor{red}{utilize os verbos em terceira pessoa do singular ou do plural, nunca em primeira pessoa.}  <- Corrigindo

A partir dos resultados apresentados nota-se que o classificador possui uma alta taxa de \textit{precision} para a classe positiva, enquanto apresenta taxas menores para a classe de comentários pejorativos, onde se encontram os termos pejorativos.

Essa diferença se dá principalmente pela quantidade de comentários negativos detectados e presentes, pois se pode ver na tabela \ref{resultados-qtd-commentarios-classes}, vídeos com público alvo infantil possuem menor quantidade de comentários negativos detectados.

Por outro lado, a pontuação de \textit{recall} para comentários negativos foi satisfatória para 4 dos 6 vídeos analisados, mostrando que dentre os comentários pejorativos de fato existentes, o classificador gerado conseguiu detectar corretamente sua maior parte. Novamente é possível notar que as menores taxas de \textit{recall} se dão nos vídeos com menos comentários e com menor quantidade de comentários negativos, vide Tabela \ref{resultados-qtd-commentarios-classes}.%criar tabela.

% não sei se posso falar algo do tipo: vídeos infantis tem menor taxa de comentários ofensivos enquanto que vídeos com público adolescente tem maior taxa de comentários ofensivos.

Logo nota-se que quanto mais comentários o vídeo apresentar, maior a taxa de acerto eventos de comentários negativos e maior a relevância da classificação realizada.
Os dados obtidos podem ser de fato considerados úteis para a classificação de comentários pejorativos de vídeos no Youtube, mais precisamente vídeos com público alvo adolescente. O que torna relevante o uso da extensão criada para o navegador da web Google Chrome.
