\newpage
\section{RESULTADOS}

Foram coletados 87.094 comentários ao todo, dentre 5 vídeos infantis e 1 vídeo com público adolescente.
A Tabela \ref{resultados-pre-comentarios} apresenta os meta dados da base adquirida: nome dos vídeos,  total de visualizações no momento da coleta, data da coleta e quantidade de comentários por vídeo.

\begin{table}[H] \footnotesize
\centering
	\Caption{\label{resultados-pre-comentarios} Vídeos selecionados e comentários obtidos}%
%META-DADOS	
\begin{tabular}{|p{5.5cm}|c|c|c|}
\hline
\textbf{Título do Vídeo} & \textbf{Visualizações} & \textbf{Data da Coleta} & \textbf{Comentários obtidos} \\ \hline
PARABÉNS DA GALINHA PINTADINHA - Clipe Música Oficial - Galinha Pintadinha DVD 4 & 388.515.807 & 20/11/2017 12:00 & 11.669 \\ \hline
Galinha Pintadinha 4 - Clipe Música Oficial - Galinha Pintadinha DVD 4 & 90.240.721 & 20/11/2017 11:30 & 2.320 \\ \hline
UPA CAVALINHO - Clipe Música Oficial - Galinha Pintadinha DVD 4 & 112.389.836 & 27/09/2017 23:38 & 2.798 \\ \hline
Pintinho Amarelinho - DVD Galinha Pintadinha & 496.289.054 & 28/09/2017 00:05 & 12.906 \\ \hline
Galinha Pintadinha 3 - Trailer - OFICIAL & 13.849.870 & 27/09/2017 11:28 & 298 \\ \hline
Não Faz Sentido! - Crepúsculo [+13] & 15.768.111 & 15/05/2018 07:14 & 57.103 \\ \hline
% MEU MELHOR AMIGO - LUCCAS NETO & 14.762.043 & 11/06/2018 02:21 & 46.173 \\ \hline
\end{tabular}

\Fonte{Elaborado pelo autor}
\end{table}

Por conta do vídeo \textbf{Não Faz Sentido! - Crepúsculo [+13]} ser o com maior quantidade de comentários e devido ao seu público alvo ser composto por adolescentes, o conjunto de treino e de testes da classificação partiu dele. 

% Tabela 3
A Tabela \ref{resultados-qtd-commentarios-classes} exibe a quantidade de comentários obtidos em cada classe, para cada vídeo analisado. Esse total é obtido através do método \textit{classification\_report()} presente na biblioteca \textit{scikit-learn}.

\begin{table}[H] \footnotesize
\centering
	\Caption{\label{resultados-qtd-commentarios-classes} Quantidade de comentários em cada classe por vídeo}%
%Comentarios positivos x negativos	
\begin{tabular}{|p{5.5cm}|c|c|}
\hline
\textbf{Título do Vídeo} & \textbf{Neutros} & \textbf{Pejorativos} \\ \hline
PARABÉNS DA GALINHA PINTADINHA - Clipe Música Oficial - Galinha Pintadinha DVD 4 & 11.152 & 517 \\ \hline
Galinha Pintadinha 4 - Clipe Música Oficial - Galinha Pintadinha DVD 4 & 2.217 & 103 \\ \hline
UPA CAVALINHO - Clipe Música Oficial - Galinha Pintadinha DVD 4 & 2.654 & 144 \\ \hline
Pintinho Amarelinho - DVD Galinha Pintadinha & 11.407 & 1.499 \\ \hline
Galinha Pintadinha 3 - Trailer - OFICIAL & 277 & 21 \\ \hline
Não Faz Sentido! - Crepúsculo [+13] & 21.066 & 11.064 \\ \hline

\end{tabular}

\Fonte{Elaborado pelo autor}
\end{table}

Como foi descrito no Capítulo 6, o modelo gerado foi utilizado em todos os comentários obtidos, com exceção dos comentários utilizados no grupo de treino, tendo sido a etapa de geração do modelo de classificação, o único passo a ser executado somente uma vez. Além disso, apesar do grupo a partir do qual foi gerado o modelo de classificação ter partido de apenas um vídeo, os comentários negativos são similares dentre todos os vídeos, seja de conteúdo infantil ou juvenil.
\begin{comment}
\textcolor{red}{Não entendi Isaias, como você calculou recall e precision dos comentários dos outros projetos, se você não os avaliou com a SentiStregth... ou avaliou? Isso não fica claro e gera algumas dúvidas. Além disso, é bom dizer que mesmo que você tenha usado somente de um tipo de video, os comentários para criar o modelo e o testar, você tem comentários similares nos outros vídeos...}
\textcolor{pink}{I: Todos foram avaliados com Sentistrength, vou deixar mais explicito aqui e também na metodologia. Sobre os comentários similares, irei mencionar.}
\end{comment}
Seguindo a metodologia de 50\% de comentários para treino e 50\% para teste do classificador, foram utilizados 32.131 comentários para treino e 32.130 comentários para teste. 

Em relação à analise de dados com conjuntos de dados de entrada desbalanceados, para um modelo de classificicação, \cite{tan2009DataMining} afirma que as métricas de \textbf{\textit{precision}} e \textbf{\textit{recall}} se sobressaem à métrica de \textit{taxa de acertos}, pois esta última não considera o peso das classes sendo analisadas.

Na Tabela \ref{resultados-precision} estão os valores de \textit{precision}, descritos por vídeo, após a classificação. Os valores são obtidos com o uso da biblioteca \textit{sklearn-metrics} \cite{scikit-learn}. A coluna \textbf{precision-pos} descreve a precisão de acerto do classificador para a classe positiva (ou neutra), enquanto a coluna \textbf{precision-neg} descreve a precisão de acerto do classficador para a classe negativa, onde estão os termos pejorativos.

Na Tabela \ref{resultados-recall} estão os valores de \textit{recall}, também descritos por vídeo e classe. A coluna \textbf{recall-pos} descreve a pontuação de \textit{recall} para a classe de comentários positivos (ou neutra), enquanto a coluna \textbf{recall-neg} descreve a pontuação de recall para a classe de comentários pejotarivos.


%\textcolor{red}{Chamar de negativo de pejorativo parece pouco intuitivo. Uma vez que desejamos saber se um vídeo contem comentários pejorativos ou não. O que você acha?} - I: Sim, me convenceu;


%\textit{Precision} pode ser definido como o número de (eventos) Positivos-Verdadeiros dividido pela soma entre Positivos-Verdadeiros e Positivos-Negativos (ou seja,aqueles que foram classificados como negativo 

Nas definições a seguir, $T_p$ (\textit{True positive} ou positivo verdadeiro) indica o número de comentários da classe de comentários positivos (neutros) que foram corretamente detectados como positivos; $F_n$ (\textit{False negative} ou falso negativo) denota a quantidade de comentários pejorativos detectados erroneamente como positivos; $F_p$ (\textit{False positive} ou falso positivo) indica a quantidade de comentários pejorativos detectados erroneamente como positivos; e $T_n$ (\textit{True negative} ou negativo verdadeiro) indica a quantidade de comentários pejorativos detectados corretamente \cite{tan2009DataMining}.

\textit{Precision} pode ser definida para a classe de comentários positivos, por exemplo, como a proporção do número de acertos pelo modelo na classe positiva pelo número de instâncias que foram classificadas como positiva pelo modelo. 

%\textcolor{red}{Por favor, explique melhor o que é $T_P$ e $F_P$, pode até buscar no livro do Tan} <- Buscando. <- Done

\begin{equation}
\label{eq:precision-descr}
 {P} = \frac{{T_p}}{{T_p}+{F_p}}
\end{equation}


Enquanto o \textit{recall}, corresponde a proporção do número de acertos, por exemplo, da classe positiva divido pelo que é realmente positivo.

\begin{equation}
\label{eq:recall-descr}
 {R} = \frac{{T_p}}{{T_p}+{F_n}}
\end{equation}

\begin{table}[H] \footnotesize
\centering
	\Caption{\label{resultados-precision} Resultados de \textit{precision}}%
	
\begin{tabular}{|p{5.5cm}|c|c|c|}
\hline
\textbf{Título do Vídeo} & \textbf{Classificados} & \textbf{precision-pos} & \textbf{precision-neg} \\ \hline
PARABÉNS DA GALINHA PINTADINHA - Clipe Música Oficial - Galinha Pintadinha DVD 4 & 11.669 & 0.98 & 0.18 \\ \hline
Galinha Pintadinha 4 - Clipe Música Oficial - Galinha Pintadinha DVD 4 & 2.320 & 0.98 & 0.14 \\ \hline
UPA CAVALINHO - Clipe Música Oficial - Galinha Pintadinha DVD 4 & 2.798 & 0.98 & 0.16 \\ \hline
Pintinho Amarelinho - DVD Galinha Pintadinha & 12.906 & 0.95 & 0.35 \\ \hline
Galinha Pintadinha 3 - Trailer - OFICIAL & 298 & 0.95 & 0.14 \\ \hline
Não Faz Sentido! - Crepúsculo [+13] & 31.130 & 0.88 & 0.71\\ \hline
\end{tabular}

\Fonte{Elaborado pelo autor}
\end{table}

\begin{table}[H] \footnotesize
\centering
	\Caption{\label{resultados-recall} Resultados de \textit{recall}}%
	
\begin{tabular}{|p{5.5cm}|c|c|c|}
\hline
\textbf{Título do Vídeo} & \textbf{Classificados} & \textbf{recall-pos} & \textbf{recall-neg} \\ \hline
PARABÉNS DA GALINHA PINTADINHA - Clipe Música Oficial - Galinha Pintadinha DVD 4 & 11.669 & 0.86 & 0.71 \\ \hline
Galinha Pintadinha 4 - Clipe Música Oficial - Galinha Pintadinha DVD 4 & 2.320 & 0.84 & 0.57 \\ \hline
UPA CAVALINHO - Clipe Música Oficial - Galinha Pintadinha DVD 4 & 2.798 & 0.81 & 0.64 \\ \hline
Pintinho Amarelinho - DVD Galinha Pintadinha & 12.906 & 0.83 & 0.67 \\ \hline
Galinha Pintadinha 3 - Trailer - OFICIAL & 298 & 0.78 & 0.48 \\ \hline
Não Faz Sentido! - Crepúsculo [+13] & 31.130 & 0.83 & 0.79\\ \hline
\end{tabular}

\Fonte{Elaborado pelo autor}
\end{table}

A partir dos resultados apresentados nota-se que o classificador possui uma alta taxa de \textit{precision} para a classe positiva, enquanto apresenta taxas menores para a classe de comentários pejorativos, onde se encontram os termos pejorativos.

Essa diferença se dá principalmente pela quantidade de comentários negativos detectados e presentes, pois pode-se notar na Tabela \ref{resultados-qtd-commentarios-classes}, vídeos com público alvo infantil possuem menor quantidade de comentários negativos detectados.

Por outro lado, a pontuação de \textit{recall} para comentários negativos foi satisfatória para 4 dos 6 vídeos analisados, mostrando que dentre os comentários pejorativos de fato existentes, o classificador gerado conseguiu detectar corretamente sua maior parte. Novamente, é possível notar que as menores taxas de \textit{recall} se dão nos vídeos com menos comentários e com menor quantidade de comentários negativos, vide Tabela \ref{resultados-qtd-commentarios-classes}.
