% resumo em inglês
\begin{resumo}[\normalfont\normalsize\bfseries\MakeUppercase Abstract]
 \begin{otherlanguage*}{english}
   
   Internet or World Wide Web is connecting more and more people each day. In Youtube, a video streaming platform and one of Internet's most popular websites, is noticeable how teenagers and kids are actively engaged in the comments section. Also noticeable are the offenses in these comments, directed to the users. When the videos are targeted to children and teenagers audience, this may negatively influence those younger ones, and it goes against Brazil's "Estatuto da Criança e do Adolescente", a constitutional law that protects children, as it says: "It should be respected the physical, moral and psychical integrity of children and teenagers alike."
   With the goal to detect if a Youtube video, targeted at kids and teenagers, has lots of offensive comments, it was built a Naïve Bayes textual classifier and a Google Chrome Browser Extension, which indicates the rates of offensive comments for the video.
   To analyze the Youtube video comments and to build the text classification model, it was used data classification and data mining tools: NLTK, scikit-learn, and Sentistrength. To help and automate the data collection and extraction steps, many scripts in Python were developed. Finally, a Google Chrome Extension was built, using web technologies, to display the resultant data for the public and to make easier to have new classifications.
   87.094 comments were collected, and 62.121 were analyzed and classified. Showing that 48.773 comments could be considered neutral and 13.348 could be considered offensive.
   It was noted that videos targeted at children did not have high offensive comments rate, while videos targeted at teenagers had a considerable rate, about 34\% for offensive comments. This means that videos targeted at teenagers tend to have more offensive comments than videos targeted at children.

   \vspace{\onelineskip}
 
   \noindent 
   \textbf{Key-words}: Information Retrieval System. Data Mining. Child and Teenagers Safeguarding.
 \end{otherlanguage*}
\end{resumo}