\documentclass[
    article,
    sumario=tradicional,
	% -- opções da classe memoir --
	12pt,				% tamanho da fonte
	openright,			% capítulos começam em pág ímpar (insere página vazia caso preciso)
	oneside,			% para impressão em verso e anverso. Oposto a oneside
	a4paper,			% tamanho do papel. 
	% -- opções da classe abntex2 --
	chapter=TITLE,		% títulos de capítulos convertidos em letras maiúsculas
	section=TITLE,		% títulos de seções convertidos em letras maiúsculas
	subsection=Title,	% títulos de subseções convertidos em letras maiúsculas
	%subsubsection=TITLE,% títulos de subsubseções convertidos em letras maiúsculas
	% -- opções do pacote babel --
	%english,			% idioma adicional para hifenização
	%french,				% idioma adicional para hifenização
	%spanish,			% idioma adicional para hifenização
	brazil				% o último idioma é o principal do documento
	]{abntex2}

\usepackage{ifthen,ifpdf}

\ifpdf
  \pdfpagewidth=\paperwidth
  \pdfpageheight=\paperheight
\fi

% --- 
% CONFIGURAÇÕES DE PACOTES
% --- 
% ---
% Pacotes básicos 
% ---
\usepackage{lmodern}			% Usa a fonte Latin Modern			
\usepackage[T1]{fontenc}		% Selecao de codigos de fonte.
\usepackage[utf8]{inputenc}		% Codificacao do documento (conversão automática dos acentos)
\usepackage{lastpage}			% Usado pela Ficha catalográfica
\usepackage{indentfirst}		% Indenta o primeiro parágrafo de cada seção.
\usepackage{color}				% Controle das cores
\usepackage{graphicx}			% Inclusão de gráficos
\usepackage{microtype} 			% para melhorias de justificação
\usepackage{ufc-abntex2}
\usepackage{enumitem}
\usepackage{amsmath}
\usepackage{booktabs}
\usepackage{multirow}
\usepackage{titlesec}
\usepackage{mathptmx}                               % Usa a fonte Times New Roman
\usepackage{float}
%formatação do Sumário
\usepackage{etoolbox}                               
% Usado para alterar a fonte da Section no Sumário
\usepackage[nogroupskip,nonumberlist,acronym]{glossaries}     \usepackage{listings}    
%\usepackage{tocloft}
\usepackage{pdfpages}

%
% ---
		
% ---
% Pacotes adicionais, usados apenas no âmbito do Modelo Canônico do abnteX2
% ---
\usepackage{lipsum}				% para geração de dummy text
% ---
\usepackage{longtable}
% ---
% Pacotes de citações
% ---
%\usepackage[brazilian,hyperpageref]{backref}	 % Paginas com as citações na bibl
%\usepackage[alf]{abntex2cite}	% Citações padrão ABNT

%Com et al nas referências
%\usepackage[alf, abnt-emphasize=bf, bibjustif, recuo=0cm, abnt-etal-cite=3, abnt-etal-text=it, abnt-etal-list=3]{abntex2cite} 

%Sem et al nas referências
\usepackage[alf, abnt-emphasize=bf, bibjustif, recuo=0cm, abnt-etal-cite=3, abnt-etal-text=it, abnt-etal-list=0]{abntex2cite} 

% Ambiente para alineas e e subalineas (incisos) com ponto
\newlist{alineascomponto}{itemize}{2}
\setlist[alineascomponto,1]{label={$\bullet$},topsep=0pt,itemsep=0pt,leftmargin=\parindent+\labelwidth-\labelsep}%
\setlist[alineascomponto,2]{label={--},topsep=0pt,itemsep=0pt,leftmargin=*}
\newlist{subalineascomponto}{enumerate}{1}
\setlist[subalineascomponto,1]{label={$\circ$},topsep=0pt,itemsep=0pt,leftmargin=*}%
% ---

% Ambiente para alineas e e subalineas (incisos) com numeros
\newlist{alineascomnumero}{enumerate}{2}
\setlist[alineascomnumero,1]{label={$\arabic*$.},topsep=0pt,itemsep=0pt,leftmargin=\parindent+\labelwidth-\labelsep}%
\setlist[alineascomnumero,2]{label={--},topsep=0pt,itemsep=0pt,leftmargin=*}
\newlist{subalineascomnumero}{enumerate}{1}
\setlist[subalineascomnumero,1]{label={$\arabic*$.},topsep=0pt,itemsep=0pt,leftmargin=*}%
% ---

% ---
% Configurações do pacote backref
% Usado sem a opção hyperpageref de backref
%\renewcommand{\backrefpagesname}{Citado na(s) página(s):~}
% Texto padrão antes do número das páginas
%\renewcommand{\backref}{}
% Define os textos da citação
%\renewcommand*{\backrefalt}[4]{
%	\ifcase #1 %
%		Nenhuma citação no texto.%
%	\or
%		Citado na página #2.%
%	\else
%		Citado #1 vezes nas páginas #2.%
%	\fi}%
% ---

% ---
% Configurações de aparência do PDF final

% alterando o aspecto da cor azul
\definecolor{blue}{RGB}{41,5,195}

% informações do PDF
\makeatletter
\hypersetup{
     	%pagebackref=true,
		pdftitle={\@title}, 
		pdfauthor={\@author},
    	pdfsubject={\imprimirpreambulo},
	    pdfcreator={LaTeX with abnTeX2},
		pdfkeywords={abnt}{latex}{abntex}{abntex2}{trabalho acadêmico}, 
		colorlinks=true,       		% false: boxed links; true: colored links
    	linkcolor=blue,          	% color of internal links
    	citecolor=blue,        		% color of links to bibliography
    	filecolor=magenta,      		% color of file links
		urlcolor=blue,
		bookmarksdepth=4
}
\makeatother
% --- 

\makeatletter
\def\@biblabel#1{}
\renewenvironment{thebibliography}[1]
     {\section*{\refname}%
      \@mkboth{\MakeUppercase\refname}{\MakeUppercase\refname}%
      \list{\@biblabel{\@arabic\c@enumiv}}%
           {\settowidth\labelwidth{\@biblabel{#1}}%
            \leftmargin\labelwidth
            %\advance\leftmargin\labelsep
            \@openbib@code
            \usecounter{enumiv}%
            \let\p@enumiv\@empty
            \renewcommand\theenumiv{\@arabic\c@enumiv}}%
      \sloppy
      \clubpenalty4000
      \@clubpenalty \clubpenalty
      \widowpenalty4000%
      \sfcode`\.\@m}
     {\def\@noitemerr
       {\@latex@warning{Empty `thebibliography' environment}}%
      \endlist}
\makeatother

% --- 
% Espaçamentos entre linhas e parágrafos 
% --- 

% O tamanho do parágrafo é dado por:
\setlength{\parindent}{1.5cm}


\setlength{\parindent}{2cm}

% ---
% compila o indice
% ---
\makeindex
% ---


% Define as margens do documento
\setlrmarginsandblock{3cm}{2cm}{*} % externa / interna
\setulmarginsandblock{3cm}{2cm}{*} % superior / inferior

% Define o espaço entre linhas para 1.5 cm
%\OnehalfSpacing	Bug no ABNTEX2?
\renewcommand{\baselinestretch}{1.5}
