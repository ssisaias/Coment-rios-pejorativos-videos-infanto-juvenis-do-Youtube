\section{PROCEDIMENTOS METODOLÓGICOS}

Para alcançar o objetivo de classificar os comentários pejorativos em vídeos infantis do Youtube, os passos a seguir serão realizados ao longo da execução do trabalho.
Uma ilustração da ordem de execução dos passos é descrita na Figura \ref{fig:metodologia}

\begin{itemize}
    \item Escolha dos vídeos
    \item Coleta de dados
    \item Pré-Processamento
    \item Definição dos critérios de classificação
    \item Treinamento do modelo de classificação
    \item Avaliação do modelo de classificação
    \item Indicar vídeos com comentários pejorativos
    %% Se der tudo certo prox fim de semana, vou colocar aqui o plugin do chrome :D
\end{itemize}

\begin{figure}[H] %use h para forçar que a figura fique abaixo do texto
	\caption{\label{fig:metodologia} Ilustração do procedimento metodológico}
	\begin{center}
	    \includegraphics[scale=0.7]{figuras/figura_6.png} % altere o atributo scale para o tamanho da figura
	\end{center}
	\legend{Fonte: autor}
%Verificar se precisa por a data aqui 
\end{figure}


% Passos sugeridos de acordo com os objetivos específicos:
% Classificar os comentários: (1) Coleta de Dados (2) Pré-Processamento (3) Definição dos Critérios de Classificação
% Analisar os comentários: (4) Treinamento do Modelo de Classificação
% Indicar os vídeos Pejorativos (5) xxxxxx
% Avaliar o modelo de classificação (6) Avaliação do Modelo de Classificação
\subsection{Escolha dos vídeos}
O principal critério para que um vídeo seja escolhido para este trabalho, é que seu público alvo seja infantil ou adolescente. Naturalmente, quanto mais comentários o vídeo possuir, maior a quantidade de dados para análise estarão disponíveis, logo o ideal é buscar vídeos infantis que possuam um número expressivo de comentários, algo em torno de pelo menos mil. Porém nem sempre é possível encontrar vídeos com essa quantidade ideal de comentários, o que não impede que também sejam utilizados. 

A popularidade do canal que postou o vídeo pode ser um fator a ser considerado, visto que um canal com maior popularidade também irá possuir vídeos com mais comentários.

\subsection{Coleta de Dados}
A fim de ter os dados para o ponto de partida da pesquisa e tendo em conta a grande quantidade de comentários em vídeos com público-alvo jovem, foi desenvolvida uma ferramenta em Python para coleta dos comentários em vídeos do Youtube. O objetivo da ferramenta é ser simples e objetiva na coleta desses comentários.

A aplicação desenvolvida pelo autor, utiliza a API do Youtube (Versão 3) e permite tanto obter os comentários de topo (\textit{top level comments} ou \textit{Comment Threads}), como as réplicas à esses comentários, dado o ID do vídeo que pode ser encontrado na sua URL, permitindo obter todos os comentários públicos disponíveis no vídeo, até o momento da coleta. O código fonte para a aplicação pode ser encontrado em: \footnote{https://github.com/ssisaias/ytCommentMiner}.


\subsection{Pré-Processamento}
Uma vez que os dados armazenados não estão em formato adequado para extração do conhecimento, faz-se necessária a aplicação de métodos para \textit{extração e integração}, \textit{transformação},
% ta mas o que é isso? ai vou preicsar olhar no referencial teórico e adaptar conforme o que eu fiz aqui!!!
\textit{limpeza}, \textit{seleção e redução} de volume desses dados, antes da etapa de mineração \cite{morais2007mineraccao}.

Inicialmente observa-se uma grande quantidade de comentários sem sentido em vídeos infantis no Youtube, compostos quase que inteiramente por espaços em branco ou símbolos e letras aleatórios. Em vídeos destinados à adolescentes, há menor ocorrência de comentários sem sentido. 

Nesta etapa um script escrito em Python é executado para extrair somente os textos dos comentários obtidos através da ferramenta de extração. Em seguida, através da ferramenta NLTK, o pré-processamento é realizado: os acentos são removidos e os textos são tokenizados, através de um processo que remove palavras sem valor semântico para o classificador e reduz palavras com valor semântico ao seu radical.

Vale notar que a API do Youtube também retorna o texto dos comentários contendo caracteres unicode, que representam letras acentuadas e alguns poucos símbolos, como apóstrofo (') e \textit{ampersand} (\&). Ao utilizar a versão 3 da linguagem Python, os caracteres unicode são automaticamente substituídos pelo caractere correspondente na língua portuguesa, podendo assim o texto continuar a ser pré-processado.


\subsection{Definição dos Critérios de Classificação}
%Até o momento a classificação será manual!
A classificação dos comentários é definida manualmente, levando em consideração comentários de amostragem, que serão classificados como \textbf{pejorativos} ou \textbf{não pejorativos}. 

Dado a enorme quantidade de comentários coletados, um dicionário de classificação também foi criado para a ferramenta \textbf{SentiStrength}, afim de facilitar a criação do grupo de treino para o modelo de classificação. % O dicionário pode ser verificado no apendice X. (colocar a referencia ao apendice aqui)

Após a classificação manual e a classificação assistida utilizando a SentiStrength, um modelo será treinado para que possa classificar automaticamente os novos comentários obtidos do Youtube.

\subsection{Treinamento do Modelo de Classificação}
O classificador Naïve Bayes deve receber um conjunto de dados de treino, e um conjunto de testes para averiguar a precisão de classificação \cite{ZhangandLi2007Bayes}. Esses conjuntos de dados serão escolhidos de forma aleatória \textbf{dentro de uma fatia} do total comentários coletados, sendo aproximadamente 50\% para treino e 50\% para teste. %O tamanho do conjunto de treino é menor dada a grande quantidade de comentários já coletados até o momento, e o fato de que estes comentários de treino serão classificados manualmente.

\subsection{Avaliação do Modelo de Classificação}
O conhecimento extraído na fase de mineração de dados pode gerar uma grande quantidade de padrões \cite{morais2007mineraccao}.

O modelo gerado é enfim testado utilizando os comentários que não foram utilizados na fase de treino. 
A avaliação do modelo é feita através da ferramenta Scikit Learn, que fornece métodos de score prontos para as medidas de \textit{precision} e \textit{recall}. %preciso mencionar elas lá em cima, quando for falar do scikit. % talvez colocar alguma referencia aqui seja bom tb

%O modelo de classificação gerado agora pode ser testado utilizando todos os comentários que foram coletados afim de chegar as conclusões sobre os tipos de comentários realizados em vídeos infantis e o grau de segurança da seção de comentários de cada vídeo.

\subsection{Indicar vídeos com comentários pejorativos}
Nessa etapa, iremos avaliar os resultados do processamento dos comentários, agregando agora pelos vídeos dos quais foram coletados, indicando quais os vídeos tem maiores taxas de comentários pejorativos e que não são considerados seguros para crianças e adolescentes.



\subsection{Criação de um plugin do Google Chrome}

Ao obter um modelo de classificação, foi desenvolvida uma extensão do navegador da web Google Chrome que se utiliza de um \textit{Web Service} que retorna a margem de comentários negativos e positivos para o vídeo sendo assistido naquele momento. 

Caso os comentários do vídeo ainda não tenham sido classificados, o serviço irá iniciar um procedimento automático de classificação, tornando os resultados disponíveis dentro de no máximo algumas horas.

A Figura \ref{fig:chrome_plugin} mostra um exemplo de classificação retornada pela extensão em um vídeo do Youtube.

\begin{figure}[H] %use h para forçar que a figura fique abaixo do texto
	\caption{\label{fig:chrome_plugin} Extensão do Google Chrome}
	\begin{center}
	    \includegraphics[scale=1.2]{figuras/extensao_chrome.PNG} % altere o atributo scale para o tamanho da figura
	\end{center}
	\legend{Fonte: Autor}
%Verificar se precisa por a data aqui 
\end{figure}

Os códigos da extensão \footnote{https://github.com/ssisaias/safe-youtube} e do serviço web \footnote{https://github.com/ssisaias/safe-youtube-service}, estão disponíveis publicamente e podem ser encontrados na plataforma Github.

\begin{comment}
\textit{A última seção dos procedimentos é o cronograma. Apresente a versão que entregará à banca ao final do semestre. Se seus procedimentos não estiverem organizados em subseções, esta será a subseção 4.1. Após ler, remova este texto explicativo.}
\end{comment}

% TCC 2 - remove cronograma
\begin{comment}
\subsection{Cronograma}

\begin{table}[H]
\centering
\resizebox{\textwidth}{!}{\begin{tabular}{|l|c|c|c|c|c|c|c|c|c|c|c|c|c|c|c|c|c|c|c|c|c|}
\hline
\multicolumn{1}{|c|}{\multirow{2}{*}{ATIVIDADES}} & \multicolumn{8}{|c|}{2017}  & \multicolumn{12}{c|}{2018}  \\ \cline{2-21} 
\multicolumn{1}{|c|}{} & 
\multicolumn{2}{c|}{Set} &
\multicolumn{2}{c|}{Out} & 
\multicolumn{2}{c|}{Nov} & 
\multicolumn{2}{c|}{Dez} & 
\multicolumn{2}{c|}{Fev} & 
\multicolumn{2}{c|}{Mar} &
\multicolumn{2}{c|}{Abr} & 
\multicolumn{2}{c|}{Mai} &
\multicolumn{2}{c|}{Jun} &
\multicolumn{2}{c|}{Jul}\\ \hline
\begin{tabular}[c]{@{}l@{}}Definição do projeto\end{tabular} & x &  &  &  &  &  &  & - & & &  &  &  &  &  &  & &  &  &\\ \hline
Coleta inicial dos comentários &  & x &  &  &  &  &  & - &  &  &  &  &  & & & & &  & &\\ \hline
Pesquisa e definição do referencial teórico &  &  & x & x &  &  &  & - &  &  &  &  &  & & & & &  &  &\\ \hline
Escrita do TCC I &  & x & x & x & x & x &  & - &  &  &  &  &  & & & & &   &  &\\ \hline
Defesa do TCC I &  &  &  &  &  &  & x & - &  &  &  &  &  & & & & &  &  &\\ \hline
Escolha dos vídeos &  &  &  &  &  &  &  & - & x &  &  &  &  & & & & &   &  &\\ \hline
Coleta de comentários do Youtube &  &  &  &  &  &  &  & - &  & x &  &  &  & & & & &   &  &\\ \hline
Pré Processamento dos comentários &  &  &  &  &  &  &  & - &  &  & x & x &  &  & & & &  &  &\\ \hline
Treino do modelo de classificação &  &  &  &  &  &  &  & - &  &  &  &  & x & x &  &  & &   &  &\\ \hline
Avaliação do modelo de classificação &  &  &  &  &  &  &  & - &  &  &  &  &  &  & x & x & x &  &  & \\ \hline
Indicar vídeos com comentários pejorativos &  &  &  &  &  &  &  & - &  &  &  &  &  &  & &  &  & x &  &\\ \hline
Defesa da Monografia &  &  &  &  &  &  &  & - &  &  &  &  &  &  & & &  &  & x &\\ \hline
\end{tabular}}
\end{table}

\end{comment}
